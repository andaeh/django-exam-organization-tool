\newcommand{\I}[1]{\ensuremath{\mathbb {#1}}}

\newcommand{\tunit}[1]{\rm #1}
\newcommand{\unit}[1]{\, \rm #1}
\newcommand{\funit}[2]{\ensuremath{\, \frac{\rm #1}{\rm #2}}}
\newcommand{\tfunit}[2]{\ensuremath{\frac{\rm #1}{\rm #2}}}
\DeclareMathOperator{\const}{const}
\DeclareMathOperator{\rg}{rang}
\DeclareMathOperator{\asin}{asin}
\DeclareMathOperator{\acos}{acos}
\DeclareMathOperator{\atan}{atan}
\DeclareMathOperator{\e}{\mathrm{e}}
\DeclareMathOperator{\TIP}{TIP}
\DeclareMathOperator{\HOP}{HOP}
\DeclareMathOperator{\WEP}{WEP}
\DeclareMathOperator{\TEP}{TEP}


\newcommand{\lorentz}{\sqrt{1-\frac{v^2}{c^2}}}
\newcommand{\dd}{\mathrm d}

\newcommand{\E}[1]{\ensuremath{\cdot10^{#1}}}
\newcommand{\g}{\ensuremath{9,81 \frac{\unit{m}}{\unit{s}^2}}}

\newcommand{\vektor}[2]{\begin{pmatrix}#1 \\#2\end{pmatrix}}

\newcommand{\ddvektor}[3]{\begin{pmatrix}#1 \\#2 \\ #3\end{pmatrix}}

%Farben
\definecolor{LightBlue}{HTML}{3333b2}
\setlength{\fboxrule}{0.3mm}

\newcommand{\titelfolie}
{
\begin{frame} 
 \centering
  \ifthenelse{\boolean{mylogo}}{
  \vbox to .75\paperheight {
  \vspace{2cm}
  \textbf{\Large \MyTitle}
  \vfill
  \includegraphics[width=.8\textwidth]{G:/03_TeX-Vorlagen/Logo_re.png}
  }
  }{
  \textbf{\Large \MyTitle}
  }
 \end{frame}
 
   \section{\MyTitle} 
}



%%%%%%%%%%%%%%%%%%%%%%%%% SA-Commands %%%%%%%%%%%%%%%%%%%%%%%%% 

\newcommand{\kopfzeile}[2]{
\begin{tabularx}{\textwidth}{|l|X|l|}
\hline 
#1 & \centering \textbf{#2} & \(\,^\text{\tiny{Datum}}\)\;\;\;\;\;\;\;\;\;\;\; \\
\hline
\end{tabularx}
}

\newcommand{\samakopfzeile}[3]{
\begin{tabularx}{0.8\textwidth}{|l|X|l|}
\hline 
#1 & \centering \textbf{#2} & #3 \\
\hline
\end{tabularx}

\vspace{3mm}
\uline{Name: \hspace{7cm}}
\vspace{3mm}
 
  \vspace{-1mm}
 \begin{minipage}{0.99\textwidth}
 \textbf{Hinweise:}
 \vspace{-2mm}
 \textit{\begin{itemize}\itemsep0pt
 		\item Der Rechenweg ist stets Teil der Lösung und muss nachvollziehbar notiert sein. 
 		\item Alle Lösungen sind in möglichst einfacher Form und komplett gekürzt anzugeben. 
 		\item Formale Fehler sowie die äußere Form können in die Bewertung mit einfließen.
 	\end{itemize}}
 
 \end{minipage}
 \vspace{3mm}
}

\newcommand{\saphkopfzeile}[3]{
\begin{tabularx}{0.80\textwidth}{|l|X|l|}
\hline 
#1 & \centering \textbf{#2} & #3 \\
\hline
\end{tabularx}

\vspace{3mm}
\uline{Name: {\color{white}NameNameNameNameNameNameNameName}}
\vspace{1mm}

 \vspace{2mm}
 \begin{minipage}{0.99\textwidth}
\textbf{Hinweise:}
\vspace{-2mm}
\begin{itemize}\itemsep0pt
	\item Achten Sie auf eine konsistente Benennung der physikalischen Größen.
	\item Geben Sie zuerst den allgemeinen Ansatz an, lösen Sie nachvollziehbar nach der gesuchten Variable auf und setzen Sie anschließend die entsprechenden Größen ein.
	\item Achten Sie auf die Anzahl gültiger Ziffern.
	\item Formale Fehler sowie die äußere Form können in die Bewertung mit einfließen.
\end{itemize}
\end{minipage}
}

\newcommand{\saphkopfzeileohi}[3]{
	\begin{tabularx}{0.80\textwidth}{|l|X|l|}
		\hline 
		{#1} & \centering \textbf{#2} & {#3} \\
		\hline
	\end{tabularx}

	\vspace{3mm}
	\uline{Name: {\color{white}NameNameNameNameNameNameNameName}}
}

\newcommand{\BEWarning}{
\begin{minipage}{0.75\textwidth}
  \centering
  \textbf{-- Alle BE-Angaben erfolgen unter Vorbehalt --}
 \end{minipage}
}

\newcommand{\nextpage}{
\vfill
\hfill $\longrightarrow$ \textbf{BITTE WENDEN}
\newpage
}

\newcounter{summe}
\setcounter{summe}{0}
\newcommand{\schlussa}[1]{
\hfill \textbf{Gesamt}
 
 \hfill \begin{tabular}{|p{1cm}@{/}R{1.5cm}|}
                                              \hline
                                              & \textbf{#1 BE} \\ 
                                              \hline
                                             \end{tabular}
 
 \vspace{5mm}
 
}

%%%%%%%%%%%%%%%%%%%%%%%%%%%% Andere Kopfzeilen %%%%%%%%%%%%%%%%%%%%%%%%%%%%%%%%%%%%%%

\newcommand{\lzkopfzeile}[1]{
\begin{tabularx}{\textwidth}{|Y|}
\hline 
\textbf{#1} \\
\hline
\end{tabularx}
}

\newcommand{\wokopfzeile}[2]{
\begin{tabularx}{\textwidth}{|l|X|l|}
\hline 
#1 & \centering \textbf{#2} & \(\,^\text{\tiny{Name}}\)\hspace{5cm} \\
\hline
\end{tabularx}
}

\newcommand{\seite}{\vspace*{-15mm}
 \begin{center}Seite \thepage{} von \pageref{LastPage}\end{center}
}

\newcommand{\punkt}{\;\;{\color{red}\Large\checkmark}}
\newcommand{\hpunkt}{\;\;{\color{red}\checkmark\kern-1.1ex\raisebox{.7ex}{\rotatebox[origin=c]{125}{--}}}}

\renewcommand{\arraystretch}{1.4}

\newcommand{\Ekin}{\ensuremath{E_\text{kin}}}
\newcommand{\Epot}{\ensuremath{E_\text{pot}}}
\newcommand{\Eges}{\ensuremath{E_\text{ges}}}

\newcommand{\ges}{\ensuremath{_\text{ges}}}

\newcommand{\enumzahlen}{\renewcommand{\labelenumi}{\arabic{task}.\arabic{enumi}}}
