\newcounter{lsg}
\newcounter{script}
\setcounter{script}{0}
\setcounter{lsg}{0} %1 = Lösung zeigen, 0 = Lösung verbergen

\newcounter{leerzeilen}
\setcounter{leerzeilen}{0}

%\newcounter{zeilen}

\newcommand{\Loes}[1]{
\ifnum\value{lsg}=0
  {\color{white}{#1}}
\else
  #1
\fi
}

\newcommand{\SaLoes}[1]{
\ifnum\value{lsg}=0
{}
\else
\begin{loesung}
  #1
\end{loesung}
\fi
}

\newcommand{\TikzLoes}[1]{
\ifnum\value{lsg}=0
 %#1
\else
 #1
\fi
}

\newcommand{\MathLoes}[3]{
\ifnum\value{lsg}=0
 \karo{#1}{#2} 

\else
\begin{loesung}
 #3
\end{loesung}
\fi
}

\newcommand{\TLoes}[2]{
\ifnum\value{lsg}=1
  \uline{#2}
\else
\whiledo {\value{leerzeilen}<#1}
{
\uline{\hfill}

\vspace{2mm}
\addtocounter{leerzeilen}{1}
}
\uline{\hfill}
\setcounter{leerzeilen}{0}
\fi
}

%Das gleiche mit anderem Befehl
\newcommand{\Luecke}[1]{
\ifnum\value{lsg}=1
  \uline{#1}
\else
    \uline{\hspace{1.5cm} {\color{white}#1} \hspace{1.5cm}}
\fi
}

\newcommand{\MLuecke}[1]{
\ifnum\value{lsg}=1
  \uline{#1}
\else
    \uline{\hspace{.5cm} {\color{white}#1} \hspace{.5cm}}
\fi
}